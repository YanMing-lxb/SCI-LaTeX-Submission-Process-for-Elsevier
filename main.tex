%
%  =======================================================================
%  ····Y88b···d88P················888b·····d888·d8b·······················
%  ·····Y88b·d88P·················8888b···d8888·Y8P·······················
%  ······Y88o88P··················88888b·d88888···························
%  ·······Y888P··8888b···88888b···888Y88888P888·888·88888b·····d88b·······
%  ········888······"88b·888·"88b·888·Y888P·888·888·888·"88b·d88P"88b·····
%  ········888···d888888·888··888·888··Y8P··888·888·888··888·888··888·····
%  ········888··888··888·888··888·888···"···888·888·888··888·Y88b·888·····
%  ········888··"Y888888·888··888·888·······888·888·888··888··"Y88888·····
%  ·······························································888·····
%  ··························································Y8b·d88P·····
%  ···························································"Y88P"······
%  =======================================================================
% 
%  -----------------------------------------------------------------------
% Author       : 焱铭
% Date         : 2023-07-01 15:54:14 +0800
% LastEditTime : 2024-09-19 21:47:15 +0800
% FilePath     : /SCI-LaTeX-Submission-Process-for-Elsevier/main.tex
% Description  : Manuscript-CN ⇒ Manuscript-CN ⇒ Major-Revision ⇒ Minor-Revision ⇒ Accepted Manuscript ⇒ Final Manuscript
%  -----------------------------------------------------------------------
%
% \special{dvipdfmx:config z 0}                                                                                % XeLaTeX取消PDF压缩,加快编译速度
% \pdfcompresslevel=0                                                                                          % PdfLaTeX取消PDF压缩,加快编译速度
% \pdfobjcompresslevel=0                                                                                       % LuaLaTeX取消PDF压缩,加快编译速度
\documentclass[preprint,5p,sort&compress,times,UTF8]{elsarticle}                                             % preprint review  % compress压缩引用序号 
%% Use the option review to obtain double line spacing
%% \documentclass[authoryear,preprint,review,12pt]{elsarticle}

%% Use the options 1p,twocolumn; 3p; 3p,twocolumn; 5p; or 5p,twocolumn
%% for a journal layout:
%% \documentclass[final,1p,times]{elsarticle}
%% \documentclass[final,1p,times,twocolumn]{elsarticle}
%% \documentclass[final,3p,times]{elsarticle}
%% \documentclass[final,3p,times,twocolumn]{elsarticle}
%% \documentclass[final,5p,times]{elsarticle}
%% \documentclass[final,5p,times,twocolumn]{elsarticle}

% -----------------------------------------------------------------------------------------------------------
% >>>>>>>>>>>>>>>>>>>>>>>>>>>>>>>>>>>>>>>>>>>>>>>> |宏包设置| <<<<<<<<<<<<<<<<<<<<<<<<<<<<<<<<<<<<<<<<<<<<<<<<
% -----------------------------------------------------------------------------------------------------------
\usepackage[cn]{tagging}                                                                                     % 通过标签进行版本切换
\usepackage{amsmath}                                                                                         % 公式输入需要 equation 以及多行公式 [fleqn] 所有公式左对齐
\usepackage{amssymb}                                                                                         % The amsthm package provides extended theorem environments
\usepackage{xcolor}                                                                                          % 修稿标注时所需颜色宏
\usepackage{lineno}                                                                                          % 该宏包可显示行号
\usepackage{microtype}                                                                                       % 用于微调字距、字间距和断行,对英文、中文等文本进行优化,使得断行更加美观和平衡

% -----------------------------------------------------
% >>>>>>>>>>>>>>>>>>> |中文文本设置| <<<<<<<<<<<<<<<<<<<
% -----------------------------------------------------
\tagged{cn}{ 
    \usepackage{ctex}                                                                                        % 中文宏包
    \usepackage{iftex}
    \ifPDFTeX
        \usepackage{soul}
        \newcommand{\gl}[1]{\textcolor{orange}{##1}}
        \typeout{<<<<< |使用 PdfLaTeX 进行编译| >>>>>}
    \else
        \ifLuaTeX
            \usepackage{emoji}
        \usepackage{luacolor,lua-ul}                                                                         % lualatex需要使用lua-ul宏包进行高亮设置
            \newcommand{\gl}[1]{\highLight{##1}}
            \typeout{<<<<< |使用 LuaLaTeX 进行编译| >>>>>}
        \else
            \ifXeTeX
                \usepackage{xeCJKfntef}                                                                      % 中文高亮设置, 导入此包支持汉字特殊下划线效果 
                \newcommand{\gl}{\CJKsout*[thickness=2.5ex,format=\color{yellow}]}                           % 新定义高亮命令\gl 用来高亮显示中文
                \typeout{<<<<< |使用 XeLaTeX 进行编译| >>>>>}
            \else
                \typeout{<<<<< |Please use one of PdfLaTex, LuaLaTeX, XeLaTeX for compilation| >>>>>}
                \errmessage{<<<<< |请使用 PdfLaTex、LuaLaTeX、XeLaTeX 之一进行编译| >>>>>}
            \fi
        \fi
    \fi
}
% <<<<<<<<<<<<<<<<<<< |中文文本设置| >>>>>>>>>>>>>>>>>>>


% -----------------------------------------------------
% >>>>>>>>>>>>>>>>>>> |表格格式设置| <<<<<<<<<<<<<<<<<<<
% -----------------------------------------------------
\usepackage[caption=false,farskip=0pt,labelfont={bf}]{subfig}                                                % 设置子图所需宏包
\usepackage{booktabs}                                                                                        % 三线表
\usepackage{array}                                                                                           % 设置表格单元宽度
\usepackage{multirow}                                                                                        % 表格行合并单元格设置
\usepackage{threeparttable}                                                                                  % 添加表格注释
                                        
\usepackage[labelfont={bf}]{caption}                                                                         % 题注字体加粗
\usepackage{setspace}                                                                                        % 使用间距宏包
\AtBeginEnvironment{tabular}{\scriptsize}                                                                    % 表格字体小五
\captionsetup[figure]{aboveskip=3pt, belowskip=3pt} 
\captionsetup[table]{aboveskip=3pt, belowskip=3pt}
% <<<<<<<<<<<<<<<<<<< |表格格式设置| >>>>>>>>>>>>>>>>>>>


% -----------------------------------------------------
% >>>>>>>>>>>>>>>>>>>> |术语表设置| <<<<<<<<<<<<<<<<<<<<
% -----------------------------------------------------
\usepackage{ifthen}
\usepackage{framed}                                                                                          % 导入 framed 宏包,用于给内容添加框架效果
\usepackage{multicol}                                                                                        % 导入 multicol 宏包,提供多栏环境
\usepackage[]{nomencl}                                                                                       % 导入 nomencl 宏包,用于生成符号说明
\usepackage{etoolbox}                                                                                        % 导入 etoolbox 宏包,提供处理 TeX 和 LaTeX 代码的工

% 术语表详细设置
\renewcommand*\nompreamble{\begin{multicols}{2}}                                                             % 设置术语表为双栏显示
\renewcommand*\nompostamble{\end{multicols}}

% 更改术语之间的垂直行间距
\newlength{\nomitemorigsep}
\setlength{\nomitemorigsep}{\nomitemsep}
\setlength{\nomitemsep}{-1.25\parsep} % Baseline skip between items

% 创建术语表分组
\renewcommand{\nomgroup}[1]{%
\ifthenelse{\equal{#1}{A}}{\vspace{6pt} \item[\textbf{Greek symbols}]}{%
\ifthenelse{\equal{#1}{B}}{\vspace{6pt} \item[\textbf{Subscripts}]}{%
\ifthenelse{\equal{#1}{C}}{\vspace{6pt} \item[\textbf{Abbreviations}]}{}
}}

\itemsep\nomitemsep                                                                                          % 应用上面设置的术语垂直行间距
}
\makenomenclature                                                                                            % 打印术语表
% <<<<<<<<<<<<<<<<<<<< |术语表设置| >>>>>>>>>>>>>>>>>>>>


% -----------------------------------------------------
% >>>>>>>>>>>>>>>>>>> |链接格式设置| <<<<<<<<<<<<<<<<<<<
% -----------------------------------------------------
\usepackage{hyperref}                                                                                        % 对目录生成链接,注:该宏包可能与其他宏包冲突,故放在所有引用的宏包之后
\hypersetup{                                                                                                 % 将链接文字带颜色
    colorlinks = true,
    citecolor=blue, 
    linkcolor=black, 
    urlcolor=black,
	bookmarksopen = true,                                                                                    % 展开书签
	bookmarksnumbered = true,                                                                                % 书签带章节编号
	pdftitle = title,                                                                                        % 标题
	pdfauthor = YanMing                                                                                      % 作者
    }
    %\setlength{\mathindent}{0pt} % 将公式的缩进调整为0
% <<<<<<<<<<<<<<<<<<< |链接格式设置| >>>>>>>>>>>>>>>>>>>


% -----------------------------------------------------
% >>>>>>>>>>>>>>>>>>> |题注格式设置| <<<<<<<<<<<<<<<<<<<
% -----------------------------------------------------
\usepackage[capitalise,nameinlink]{cleveref}                                                                 % 引用宏包 ,该宏包要在 hyperref后引用   
                                                                                                             % capitalise 将引用的单词首字母大写
                                                                                                             % nameinlink 将引用的名称包含在超链接中
                                                                                                             % noabbrev 使用完整的引用名称,而不使用缩写

% 关于题注的设置
\captionsetup[subfigure]{subrefformat=simple,labelformat=simple,listofformat=subsimple}
\renewcommand\thesubfigure{(\alph{subfigure})}                                                               % 该代码与上一行 设置引用前缀子图序号带圆括号
\renewcommand{\figurename}{Fig.}                                                                             % 题注中使默认的图名字改为图显示
\renewcommand{\tablename}{Table}                                                                             % 题注中使默认的表名字改为表显示
% <<<<<<<<<<<<<<<<<<< |题注格式设置| >>>>>>>>>>>>>>>>>>>


% -----------------------------------------------------
% >>>>>>>>>>>>>>>>>>> |额外命令设置| <<<<<<<<<<<<<<<<<<<
% -----------------------------------------------------
\graphicspath{{Pictures/},}  % 图片所在文件夹, 可放置多个文件夹, 用,分隔。
% 定义新命令\Nomenclature用于包含术语表的内容
\newcommand{\Nomenclature}{
    \iftagged{cn}
    {%
%  =======================================================================
%  ····Y88b···d88P················888b·····d888·d8b·······················
%  ·····Y88b·d88P·················8888b···d8888·Y8P·······················
%  ······Y88o88P··················88888b·d88888···························
%  ·······Y888P··8888b···88888b···888Y88888P888·888·88888b·····d88b·······
%  ········888······"88b·888·"88b·888·Y888P·888·888·888·"88b·d88P"88b·····
%  ········888···d888888·888··888·888··Y8P··888·888·888··888·888··888·····
%  ········888··888··888·888··888·888···"···888·888·888··888·Y88b·888·····
%  ········888··"Y888888·888··888·888·······888·888·888··888··"Y88888·····
%  ·······························································888·····
%  ··························································Y8b·d88P·····
%  ···························································"Y88P"······
%  =======================================================================
% 
%  -----------------------------------------------------------------------
% Author       : 焱铭
% Date         : 2023-07-04 21:34:35 +0800
% LastEditTime : 2023-07-04 21:43:54 +0800
% Github       : https://github.com/YanMing-lxb/
% FilePath     : \multi-objective_optimization_microchannel_heat_sink-with_embedded_module_with_ribs_and_pin-fins\Section\Nomenclature.tex
% Description  : 
%  -----------------------------------------------------------------------
%

\begin{table*}[!t]%!t % 该方法不适用于超过一页的符号说明

    \begin{framed}
        \begin{spacing}{1}
            \nomenclature[C,01]{LED}{light-emitting diod}
            \nomenclature[C,02]{LTCC}{Low temperature cofired ceramic}
            \nomenclature[C,03]{MATD}{mean absolute temperature deviation}
            \nomenclature[C,04]{MCHS}{microchannel heat sink}
            \nomenclature[C,05]{MCHS-SR}{straight rectangular microchannel}
            \nomenclature[C,06]{MCHS-REM}{microchannels with rib embedded module}
            \nomenclature[C,07]{MCHS-PFEM}{microchannels with pin-fin embedded module}
            \nomenclature[C,08]{MCHS-RPFEM}{microchannel heat sink with embedded module with rib and pin-fin}
            \nomenclature[]{$A$}{area on the chip's upper surface $(m^2)$}
            \nomenclature[]{$W_{rib}$}{width of the rib $(\mu m)$}
            \nomenclature[]{$H_{rib}$}{height of the rib $(\mu m)$}
            \nomenclature[]{$L_{rib}$}{length of the rib $(\mu m)$}
            \nomenclature[]{$d_{rib}$}{distance between the rib $(\mu m)$}
            \nomenclature[]{$S_{pf}$}{Side length of the pin-fin $(\mu m)$}
            \nomenclature[]{$H_{pf}$}{height of the pin-fin $(\mu m)$}
            \nomenclature[]{$Re$}{Reynolds number}
            \nomenclature[]{$N_{oc}$}{number of auxiliary channels}
            \nomenclature[]{$N_{mc}$}{number of main channels}



            \nomenclature[A]{$\rho_{f}$}{density of fluid $(kg/m^3)$}
            \nomenclature[A]{$\mu_{f}$}{dynamic viscosity of fluid $(kg/(m \cdot s))$}
            \nomenclature[A]{$\Lambda$}{molecule Knudsen number of water $(m)$}
            \nomenclature[A]{$\theta$}{MATD (Mean Absolute Temperature Deviation) $(K)$}
            \nomenclature[A]{$\alpha$}{Relative rib height}
            \nomenclature[A]{$\beta$}{Relative pin-fin height}
            \nomenclature[A]{$\gamma $}{Relative number of auxiliary channels}

            \nomenclature[B]{$s$}{solid}
            \nomenclature[B]{$f$}{fluid}
            \nomenclature[B]{$c$}{chip}
            \nomenclature[B]{$rib$}{rib}
            \nomenclature[B]{$pf$}{pin-fin}
            \nomenclature[B]{$ch$}{channel}
            \nomenclature[B]{$mc$}{main channels}
            \nomenclature[B]{$oc$}{auxiliary channels}

            \nomenclature[B]{$tot$}{total value}
            \nomenclature[B]{$ave$}{average value}
            \nomenclature[B]{$max$}{maximum value}

            \nomenclature[B]{$in$}{inlet}
            \nomenclature[B]{$env$}{environmental}

            \printnomenclature[23mm]

        \end{spacing}

    \end{framed}

\end{table*}}
    {%
%  =======================================================================
%  ····Y88b···d88P················888b·····d888·d8b·······················
%  ·····Y88b·d88P·················8888b···d8888·Y8P·······················
%  ······Y88o88P··················88888b·d88888···························
%  ·······Y888P··8888b···88888b···888Y88888P888·888·88888b·····d88b·······
%  ········888······"88b·888·"88b·888·Y888P·888·888·888·"88b·d88P"88b·····
%  ········888···d888888·888··888·888··Y8P··888·888·888··888·888··888·····
%  ········888··888··888·888··888·888···"···888·888·888··888·Y88b·888·····
%  ········888··"Y888888·888··888·888·······888·888·888··888··"Y88888·····
%  ·······························································888·····
%  ··························································Y8b·d88P·····
%  ···························································"Y88P"······
%  =======================================================================
% 
%  -----------------------------------------------------------------------
% Author       : 焱铭
% Date         : 2023-07-04 21:34:35 +0800
% LastEditTime : 2023-07-04 21:43:54 +0800
% Github       : https://github.com/YanMing-lxb/
% FilePath     : \multi-objective_optimization_microchannel_heat_sink-with_embedded_module_with_ribs_and_pin-fins\Section\Nomenclature.tex
% Description  : 
%  -----------------------------------------------------------------------
%

\begin{table*}[!t]%!t % 该方法不适用于超过一页的符号说明

    \begin{framed}
        \begin{spacing}{1}
            \nomenclature[C,01]{LED}{light-emitting diod}
            \nomenclature[C,02]{LTCC}{Low temperature cofired ceramic}
            \nomenclature[C,03]{MATD}{mean absolute temperature deviation}
            \nomenclature[C,04]{MCHS}{microchannel heat sink}
            \nomenclature[C,05]{MCHS-SR}{straight rectangular microchannel}
            \nomenclature[C,06]{MCHS-REM}{microchannels with rib embedded module}
            \nomenclature[C,07]{MCHS-PFEM}{microchannels with pin-fin embedded module}
            \nomenclature[C,08]{MCHS-RPFEM}{microchannel heat sink with embedded module with rib and pin-fin}
            \nomenclature[]{$A$}{area on the chip's upper surface $(m^2)$}
            \nomenclature[]{$W_{rib}$}{width of the rib $(\mu m)$}
            \nomenclature[]{$H_{rib}$}{height of the rib $(\mu m)$}
            \nomenclature[]{$L_{rib}$}{length of the rib $(\mu m)$}
            \nomenclature[]{$d_{rib}$}{distance between the rib $(\mu m)$}
            \nomenclature[]{$S_{pf}$}{Side length of the pin-fin $(\mu m)$}
            \nomenclature[]{$H_{pf}$}{height of the pin-fin $(\mu m)$}
            \nomenclature[]{$Re$}{Reynolds number}
            \nomenclature[]{$N_{oc}$}{number of auxiliary channels}
            \nomenclature[]{$N_{mc}$}{number of main channels}



            \nomenclature[A]{$\rho_{f}$}{density of fluid $(kg/m^3)$}
            \nomenclature[A]{$\mu_{f}$}{dynamic viscosity of fluid $(kg/(m \cdot s))$}
            \nomenclature[A]{$\Lambda$}{molecule Knudsen number of water $(m)$}
            \nomenclature[A]{$\theta$}{MATD (Mean Absolute Temperature Deviation) $(K)$}
            \nomenclature[A]{$\alpha$}{Relative rib height}
            \nomenclature[A]{$\beta$}{Relative pin-fin height}
            \nomenclature[A]{$\gamma $}{Relative number of auxiliary channels}

            \nomenclature[B]{$s$}{solid}
            \nomenclature[B]{$f$}{fluid}
            \nomenclature[B]{$c$}{chip}
            \nomenclature[B]{$rib$}{rib}
            \nomenclature[B]{$pf$}{pin-fin}
            \nomenclature[B]{$ch$}{channel}
            \nomenclature[B]{$mc$}{main channels}
            \nomenclature[B]{$oc$}{auxiliary channels}

            \nomenclature[B]{$tot$}{total value}
            \nomenclature[B]{$ave$}{average value}
            \nomenclature[B]{$max$}{maximum value}

            \nomenclature[B]{$in$}{inlet}
            \nomenclature[B]{$env$}{environmental}

            \printnomenclature[23mm]

        \end{spacing}

    \end{framed}

\end{table*}}
    }
\newcommand{\rn}[1]{\uppercase\expandafter{\romannumeral#1}} % 定义大写罗马数字                                        % 定义新命令\Nomenclature用于包含术语表的内容
% <<<<<<<<<<<<<<<<<<< |额外命令设置| >>>>>>>>>>>>>>>>>>>

% <<<<<<<<<<<<<<<<<<<<<<<<<<<<<<<<<<<<<<<<<<<<<<<< |宏包设置| >>>>>>>>>>>>>>>>>>>>>>>>>>>>>>>>>>>>>>>>>>>>>>>>


% -----------------------------------------------------
%                        目标期刊
% -----------------------------------------------------
\journal{Applied Thermal Engineering}                                                                        % 输入要投稿的期刊
% ------------------------------------------------------------------------------------------------------------
%                                                  正文内容
% ------------------------------------------------------------------------------------------------------------
\begin{document}
% \linenumbers                                                                                               % 增加行号 提交修改稿时用
% \pagecolor[HTML]{e6fcf5}                                                                                   % 设置纸张背景颜色

% -----------------------------------------------------
%                        封面内容
% -----------------------------------------------------
\begin{frontmatter}
    \title{Paper Title}                                                                                      % 文章标题

    \author[rvt]{Yan Ming\corref{cor2}}
    \author[rvt]{Yan Ming\corref{cor2}}
    \author[rvt]{Yan Ming\corref{cor1}}%\fnref{fn1}
    \ead{your-email@guet.edu.cn}
    \author[rvt]{Yan Ming}%\fnref{fn1}


    \cortext [cor1]{Corresponding author}                                                                     % 通讯作者角标
    \cortext [cor2]{These authors contributed to the work equllly and should be regarded as co-first authors.}% 共同一作声明,没有可去掉
    \fntext[fn1]{This is the specimen author footnote.}                                                       % 关于作者的介绍之类,没有可去掉
    \affiliation[rvt]{organization={Guilin University Of Electronic Technology},                              % 所在学院和学校
        addressline={No.1 Jinji Road, Qixing District},
        city={Guilin},
        postcode={541004},
        state={the Guangxi Zhuang Autonomous Region},
        country={China}}

    \begin{abstract}                                                                                          % abstract的内容不要为空,否则参考文献引用不会显示
        \begin{taggedblock}{cn}
            % \gl{第一句 交代背景}
            为了解决低温共烧陶瓷(LTCC)基板的热管理问题,我们此前提出了一种带肋条和针脚嵌入式模块的微通道散热器(MC-RPFEM),并对其水热性能进行了研究。
        \end{taggedblock}

        \begin{taggedblock}{en}
            We previously proposed a microchannel heat sink with embedded module with ribs and pin-fins (MC-RPFEM) as a solution to address the thermal management issues of low-temperature co-fired ceramic (LTCC) substrates, and investigate its hydrothermal performance.
        \end{taggedblock}
    \end{abstract}
    

    % 图片摘要设置
    \begin{graphicalabstract}
        \centering
        \includegraphics*[width=1 \textwidth]{Graphical-Abstract.jpg}
    \end{graphicalabstract}

    \begin{highlights}
        \begin{taggedblock}{cn}
            \item 提出了一种优化方法来优化MC-RPFEM;
            \item 对热阻, 温度标准差, 压降进行多目标优化;
        \end{taggedblock}

        \begin{taggedblock}{en}
            \item An optimization method is proposed to optimize MC-RPFEM;
            \item Multi-objective optimization of $R_{th}$, STD, and $\Delta P$;
        \end{taggedblock}
    \end{highlights}
    
    \begin{keyword}
        Microchannel \sep Embedded \sep LTCC \sep Response surface methodology \sep NSGA-\rn{2}
    \end{keyword}
\end{frontmatter}


\begin{taggedblock}{cn}
    %
%  =======================================================================
%  ····Y88b···d88P················888b·····d888·d8b·······················
%  ·····Y88b·d88P·················8888b···d8888·Y8P·······················
%  ······Y88o88P··················88888b·d88888···························
%  ·······Y888P··8888b···88888b···888Y88888P888·888·88888b·····d88b·······
%  ········888······"88b·888·"88b·888·Y888P·888·888·888·"88b·d88P"88b·····
%  ········888···d888888·888··888·888··Y8P··888·888·888··888·888··888·····
%  ········888··888··888·888··888·888···"···888·888·888··888·Y88b·888·····
%  ········888··"Y888888·888··888·888·······888·888·888··888··"Y88888·····
%  ·······························································888·····
%  ··························································Y8b·d88P·····
%  ···························································"Y88P"······
%  =======================================================================
% 
%  -----------------------------------------------------------------------
% Author       : 焱铭
% Date         : 2023-07-04 20:56:59 +0800
% LastEditTime : 2023-07-05 13:41:22 +0800
% FilePath     : \SCI-LaTeX-Submission-Process-for-Elsevier\2-Manuscript-EN\Section\Section1.tex
% Description  : 
%  -----------------------------------------------------------------------
%

\section{Introduction}
\gl{当前现状}

随着第五代移动通信等技术的发展\cite{Lau_2022},推动了射频技术向高速、小型化和多功能方向的快速发展,电子芯片的热通量也在增加。
\Nomenclature
芯片产生的高温和芯片附近较大的温度梯度会降低集成电路的功耗。
    %
%  =======================================================================
%  ····Y88b···d88P················888b·····d888·d8b·······················
%  ·····Y88b·d88P·················8888b···d8888·Y8P·······················
%  ······Y88o88P··················88888b·d88888···························
%  ·······Y888P··8888b···88888b···888Y88888P888·888·88888b·····d88b·······
%  ········888······"88b·888·"88b·888·Y888P·888·888·888·"88b·d88P"88b·····
%  ········888···d888888·888··888·888··Y8P··888·888·888··888·888··888·····
%  ········888··888··888·888··888·888···"···888·888·888··888·Y88b·888·····
%  ········888··"Y888888·888··888·888·······888·888·888··888··"Y88888·····
%  ·······························································888·····
%  ··························································Y8b·d88P·····
%  ···························································"Y88P"······
%  =======================================================================
% 
%  -----------------------------------------------------------------------
% Author       : 焱铭
% Date         : 2023-07-04 20:57:38 +0800
% LastEditTime : 2024-09-19 22:02:26 +0800
% Github       : https://github.com/YanMing-lxb/
% FilePath     : /SCI-LaTeX-Submission-Process-for-Elsevier/Section/cn00/Section2.tex
% Description  : 
%  -----------------------------------------------------------------------
%

\section{Mathematical modeling of the microchannel heat sink}

\cref{fig:structure} shows the schematic design of MCHS-SR and the proposed MCHS-RPFEM.


\begin{figure*}[htbp] % figure* 可进行跨栏
    \centering % 居中
    \scriptsize % 设置字体
    \includegraphics[width=1 \textwidth]{Graphical-Abstract.jpg} % 
    \caption{
        (a) Straight rectangular microchannel (MCHS-SR),
        (b) microchannel heat sink with embedded modules with ribs and pin-fins (MCHS-RPFEM),
        (c) schematic diagram of the structure of MCHS-RPFEM.}
    \label{fig:structure}
\end{figure*}
Embedded module with ribs and pin-fins embedded in microchannel heat sink below chip.
\cref{tab:structure-parameter} shows the geometric parameters of the embedded module.
To investigate the effect of ribs and pin-fins on fluid flow and heat transfer on the embedded module in MCHS-RPFEM, three parameters were selected to be varied.
These three parameters are relative rib height ($\alpha$), relative pin-fin height ($\beta$), and relative number of auxiliary channels ($\gamma$).

\begin{table}[htbp]
    \centering
    \scriptsize
    \caption{MCHS-RPFEM geometric parameter table}
    \begin{tabular}{lccccccc}
        \toprule
        Geometrical parameters & $W_{rib}$ & $H_{rib}$ & $d_{rib}$ & $S_{pf}$ & $H_{pf}$ & $H_{ch}$ \\
        \midrule
        Value, $\mu m$         & 400       & 1000      & 1600      & 300      & 1000     & 1000     \\
        \bottomrule
    \end{tabular}
    \label{tab:structure-parameter}
\end{table}
    %
%  =======================================================================
%  ····Y88b···d88P················888b·····d888·d8b·······················
%  ·····Y88b·d88P·················8888b···d8888·Y8P·······················
%  ······Y88o88P··················88888b·d88888···························
%  ·······Y888P··8888b···88888b···888Y88888P888·888·88888b·····d88b·······
%  ········888······"88b·888·"88b·888·Y888P·888·888·888·"88b·d88P"88b·····
%  ········888···d888888·888··888·888··Y8P··888·888·888··888·888··888·····
%  ········888··888··888·888··888·888···"···888·888·888··888·Y88b·888·····
%  ········888··"Y888888·888··888·888·······888·888·888··888··"Y88888·····
%  ·······························································888·····
%  ··························································Y8b·d88P·····
%  ···························································"Y88P"······
%  =======================================================================
% 
%  -----------------------------------------------------------------------
% Author       : 焱铭
% Date         : 2023-07-04 20:57:50 +0800
% LastEditTime : 2023-07-04 21:44:28 +0800
% Github       : https://github.com/YanMing-lxb/
% FilePath     : \multi-objective_optimization_microchannel_heat_sink-with_embedded_module_with_ribs_and_pin-fins\Section\Section3.tex
% Description  : 
%  -----------------------------------------------------------------------
%

\section{Numerical method}

In this study, the following assumptions were made to simplify the numerical model

\begin{enumerate}[1.] % 可选类型 a) (i) Step 1.
    \item Newtonian fluid flow and steady laminar flow are used, and the fluid follows the Hagen-Poiseuille equation.
    \item The walls of the channel are rigid.
    \item Neglecting the effects of interaction forces, viscous heat, surface tension, and radiative heat transfer.
\end{enumerate}

\subsection{Governing equations and boundary conditions}

The governing equations are as follow:

Continuity equation:
\begin{equation}
    \nabla \cdot \left(\rho_f \vec{u}\right)=0
\end{equation}

Momentum equation:
\begin{equation}
    \vec{u} \cdot \nabla\left(\rho_f \vec{u}\right)=-\nabla p+\nabla \cdot\left(\mu_f \nabla \vec{u}\right)
\end{equation}

Energy equation for the fluid domain:
\begin{equation}
    \vec{u} \cdot \nabla\left(\rho_f C_{f} T_f\right)=\nabla \cdot\left(k_f \nabla T_f\right)
\end{equation}

Energy conservation equation for the solid domain:
\begin{equation}
    \nabla\left(k_s \nabla T_s\right)=0
\end{equation}


\begin{figure*}[htb] % 将长公式放入figure* 环境中进行跨栏显示
    \begin{align}
        \rho_{f}(T)= & 999.9+9.561 \times 10^{-2} T-1.013 \times 10^{-2} T^{2}+8.459 \times 10^{-5} T^{3}-3.496 \times 10^{-7} T^{4}                      \\ \notag\\
        C_{f}(T)=    & 4217-3.452 T+1.155 \times 10^{-1} T^{2}-1.862 \times 10^{-3} T^{3}+1.538 \times10^{-5}T^{4}-4.850 \times 10^{-8} T^{5}             \\ \notag\\
        k_{f}(T)=    & 5.698 \times 10^{-1}+1.772 \times 10^{-3} T-4.870 \times 10^{-6} T^{2}-2.915 \times10^{-8} T^{3}+1.094 \times 10^{-10} T^{4}       \\ \notag\\
        \mu_{f}(T)=  & 1.750 \times 10^{-3}-5.558 \times 10^{-5} T+1.172 \times 10^{-6} T^{2}-1.579 \times10^{-8} T^{3}+1.169 \times 10^{-10} T^{4}\notag \\
                     & -3.535 \times 10^{-13} T^{5}
    \end{align}
\end{figure*}


where T is the temperature ($^{\circ}C$)







\subsection{Data reduction}


\subsection{Grid independence}

    %
%  =======================================================================
%  ····Y88b···d88P················888b·····d888·d8b·······················
%  ·····Y88b·d88P·················8888b···d8888·Y8P·······················
%  ······Y88o88P··················88888b·d88888···························
%  ·······Y888P··8888b···88888b···888Y88888P888·888·88888b·····d88b·······
%  ········888······"88b·888·"88b·888·Y888P·888·888·888·"88b·d88P"88b·····
%  ········888···d888888·888··888·888··Y8P··888·888·888··888·888··888·····
%  ········888··888··888·888··888·888···"···888·888·888··888·Y88b·888·····
%  ········888··"Y888888·888··888·888·······888·888·888··888··"Y88888·····
%  ·······························································888·····
%  ··························································Y8b·d88P·····
%  ···························································"Y88P"······
%  =======================================================================
% 
%  -----------------------------------------------------------------------
% Author       : 焱铭
% Date         : 2023-07-04 21:20:55 +0800
% LastEditTime : 2023-07-04 21:45:23 +0800
% Github       : https://github.com/YanMing-lxb/
% FilePath     : \multi-objective_optimization_microchannel_heat_sink-with_embedded_module_with_ribs_and_pin-fins\Section\Section4.tex
% Description  : 
%  -----------------------------------------------------------------------
%

\section{Results and discussion}

\subsection{Numerical validations}


\subsection{Effect of geometric prameters on hydrothermal performance}

In order to explore the influence of the geometric parameters of MCHS-RPFEM on the flow and heat transfer performance, several parameters were selected for research: the relative rib height ($\alpha$), the relative pin-fin height ($\beta$), and the relative number of auxiliary channels ($\gamma$).




\subsubsection{The effect of relative rib height  ($\alpha$)}
The ratio of rib height to channel height is defined as relative rib height:




\subsubsection{Performance analysis}
    \input{Section/cn00/Section5.tex} 
\end{taggedblock}

\begin{taggedblock}{en}
    %
%  =======================================================================
%  ····Y88b···d88P················888b·····d888·d8b·······················
%  ·····Y88b·d88P·················8888b···d8888·Y8P·······················
%  ······Y88o88P··················88888b·d88888···························
%  ·······Y888P··8888b···88888b···888Y88888P888·888·88888b·····d88b·······
%  ········888······"88b·888·"88b·888·Y888P·888·888·888·"88b·d88P"88b·····
%  ········888···d888888·888··888·888··Y8P··888·888·888··888·888··888·····
%  ········888··888··888·888··888·888···"···888·888·888··888·Y88b·888·····
%  ········888··"Y888888·888··888·888·······888·888·888··888··"Y88888·····
%  ·······························································888·····
%  ··························································Y8b·d88P·····
%  ···························································"Y88P"······
%  =======================================================================
% 
%  -----------------------------------------------------------------------
% Author       : 焱铭
% Date         : 2023-07-04 20:56:59 +0800
% LastEditTime : 2023-07-05 13:41:22 +0800
% FilePath     : \SCI-LaTeX-Submission-Process-for-Elsevier\2-Manuscript-EN\Section\Section1.tex
% Description  : 
%  -----------------------------------------------------------------------
%

\section{Introduction}


With the development of technologies such as fifth generation mobile communication \cite{Lau_2022}, which is driving the rapid development of radio frequency technology toward high-speed, miniaturization \cite{Matsuzawa_2002,Miao.Jin.ea_2013}, and multifunctionality, the heat flux of electronic chips is also increasing.
\Nomenclature
The high temperature generated by the chip and the large temperature gradient near the chip will reduce the lifetime of integrated circuits.
    %
%  =======================================================================
%  ····Y88b···d88P················888b·····d888·d8b·······················
%  ·····Y88b·d88P·················8888b···d8888·Y8P·······················
%  ······Y88o88P··················88888b·d88888···························
%  ·······Y888P··8888b···88888b···888Y88888P888·888·88888b·····d88b·······
%  ········888······"88b·888·"88b·888·Y888P·888·888·888·"88b·d88P"88b·····
%  ········888···d888888·888··888·888··Y8P··888·888·888··888·888··888·····
%  ········888··888··888·888··888·888···"···888·888·888··888·Y88b·888·····
%  ········888··"Y888888·888··888·888·······888·888·888··888··"Y88888·····
%  ·······························································888·····
%  ··························································Y8b·d88P·····
%  ···························································"Y88P"······
%  =======================================================================
% 
%  -----------------------------------------------------------------------
% Author       : 焱铭
% Date         : 2023-07-04 20:57:38 +0800
% LastEditTime : 2024-09-19 22:02:26 +0800
% Github       : https://github.com/YanMing-lxb/
% FilePath     : /SCI-LaTeX-Submission-Process-for-Elsevier/Section/cn00/Section2.tex
% Description  : 
%  -----------------------------------------------------------------------
%

\section{Mathematical modeling of the microchannel heat sink}

\cref{fig:structure} shows the schematic design of MCHS-SR and the proposed MCHS-RPFEM.


\begin{figure*}[htbp] % figure* 可进行跨栏
    \centering % 居中
    \scriptsize % 设置字体
    \includegraphics[width=1 \textwidth]{Graphical-Abstract.jpg} % 
    \caption{
        (a) Straight rectangular microchannel (MCHS-SR),
        (b) microchannel heat sink with embedded modules with ribs and pin-fins (MCHS-RPFEM),
        (c) schematic diagram of the structure of MCHS-RPFEM.}
    \label{fig:structure}
\end{figure*}
Embedded module with ribs and pin-fins embedded in microchannel heat sink below chip.
\cref{tab:structure-parameter} shows the geometric parameters of the embedded module.
To investigate the effect of ribs and pin-fins on fluid flow and heat transfer on the embedded module in MCHS-RPFEM, three parameters were selected to be varied.
These three parameters are relative rib height ($\alpha$), relative pin-fin height ($\beta$), and relative number of auxiliary channels ($\gamma$).

\begin{table}[htbp]
    \centering
    \scriptsize
    \caption{MCHS-RPFEM geometric parameter table}
    \begin{tabular}{lccccccc}
        \toprule
        Geometrical parameters & $W_{rib}$ & $H_{rib}$ & $d_{rib}$ & $S_{pf}$ & $H_{pf}$ & $H_{ch}$ \\
        \midrule
        Value, $\mu m$         & 400       & 1000      & 1600      & 300      & 1000     & 1000     \\
        \bottomrule
    \end{tabular}
    \label{tab:structure-parameter}
\end{table}
    %
%  =======================================================================
%  ····Y88b···d88P················888b·····d888·d8b·······················
%  ·····Y88b·d88P·················8888b···d8888·Y8P·······················
%  ······Y88o88P··················88888b·d88888···························
%  ·······Y888P··8888b···88888b···888Y88888P888·888·88888b·····d88b·······
%  ········888······"88b·888·"88b·888·Y888P·888·888·888·"88b·d88P"88b·····
%  ········888···d888888·888··888·888··Y8P··888·888·888··888·888··888·····
%  ········888··888··888·888··888·888···"···888·888·888··888·Y88b·888·····
%  ········888··"Y888888·888··888·888·······888·888·888··888··"Y88888·····
%  ·······························································888·····
%  ··························································Y8b·d88P·····
%  ···························································"Y88P"······
%  =======================================================================
% 
%  -----------------------------------------------------------------------
% Author       : 焱铭
% Date         : 2023-07-04 20:57:50 +0800
% LastEditTime : 2023-07-04 21:44:28 +0800
% Github       : https://github.com/YanMing-lxb/
% FilePath     : \multi-objective_optimization_microchannel_heat_sink-with_embedded_module_with_ribs_and_pin-fins\Section\Section3.tex
% Description  : 
%  -----------------------------------------------------------------------
%

\section{Numerical method}

In this study, the following assumptions were made to simplify the numerical model

\begin{enumerate}[1.] % 可选类型 a) (i) Step 1.
    \item Newtonian fluid flow and steady laminar flow are used, and the fluid follows the Hagen-Poiseuille equation.
    \item The walls of the channel are rigid.
    \item Neglecting the effects of interaction forces, viscous heat, surface tension, and radiative heat transfer.
\end{enumerate}

\subsection{Governing equations and boundary conditions}

The governing equations are as follow:

Continuity equation:
\begin{equation}
    \nabla \cdot \left(\rho_f \vec{u}\right)=0
\end{equation}

Momentum equation:
\begin{equation}
    \vec{u} \cdot \nabla\left(\rho_f \vec{u}\right)=-\nabla p+\nabla \cdot\left(\mu_f \nabla \vec{u}\right)
\end{equation}

Energy equation for the fluid domain:
\begin{equation}
    \vec{u} \cdot \nabla\left(\rho_f C_{f} T_f\right)=\nabla \cdot\left(k_f \nabla T_f\right)
\end{equation}

Energy conservation equation for the solid domain:
\begin{equation}
    \nabla\left(k_s \nabla T_s\right)=0
\end{equation}


\begin{figure*}[htb] % 将长公式放入figure* 环境中进行跨栏显示
    \begin{align}
        \rho_{f}(T)= & 999.9+9.561 \times 10^{-2} T-1.013 \times 10^{-2} T^{2}+8.459 \times 10^{-5} T^{3}-3.496 \times 10^{-7} T^{4}                      \\ \notag\\
        C_{f}(T)=    & 4217-3.452 T+1.155 \times 10^{-1} T^{2}-1.862 \times 10^{-3} T^{3}+1.538 \times10^{-5}T^{4}-4.850 \times 10^{-8} T^{5}             \\ \notag\\
        k_{f}(T)=    & 5.698 \times 10^{-1}+1.772 \times 10^{-3} T-4.870 \times 10^{-6} T^{2}-2.915 \times10^{-8} T^{3}+1.094 \times 10^{-10} T^{4}       \\ \notag\\
        \mu_{f}(T)=  & 1.750 \times 10^{-3}-5.558 \times 10^{-5} T+1.172 \times 10^{-6} T^{2}-1.579 \times10^{-8} T^{3}+1.169 \times 10^{-10} T^{4}\notag \\
                     & -3.535 \times 10^{-13} T^{5}
    \end{align}
\end{figure*}


where T is the temperature ($^{\circ}C$)







\subsection{Data reduction}


\subsection{Grid independence}

    %
%  =======================================================================
%  ····Y88b···d88P················888b·····d888·d8b·······················
%  ·····Y88b·d88P·················8888b···d8888·Y8P·······················
%  ······Y88o88P··················88888b·d88888···························
%  ·······Y888P··8888b···88888b···888Y88888P888·888·88888b·····d88b·······
%  ········888······"88b·888·"88b·888·Y888P·888·888·888·"88b·d88P"88b·····
%  ········888···d888888·888··888·888··Y8P··888·888·888··888·888··888·····
%  ········888··888··888·888··888·888···"···888·888·888··888·Y88b·888·····
%  ········888··"Y888888·888··888·888·······888·888·888··888··"Y88888·····
%  ·······························································888·····
%  ··························································Y8b·d88P·····
%  ···························································"Y88P"······
%  =======================================================================
% 
%  -----------------------------------------------------------------------
% Author       : 焱铭
% Date         : 2023-07-04 21:20:55 +0800
% LastEditTime : 2023-07-05 13:40:59 +0800
% Github       : https://github.com/YanMing-lxb/
% FilePath     : \SCI-LaTeX-Submission-Process-for-Elsevier\2-Manuscript-EN\Section\Section4.tex
% Description  : 
%  -----------------------------------------------------------------------
%


\section{Results and discussion}

\subsection{Numerical validations}
To verify the accuracy of this simulation scheme, the numerical results are compared with the experimental data of several experiments \cite{Zhang.Wu.ea_2022,Qu.Mudawar_2002,Yang.Wang.ea_2017}, as shown in \cref{fig:Verification}.
\begin{figure*}[htbp]
    \centering
    \scriptsize% 设置字体大小
    \subfloat{
        \label{fig:Zhang}
        \includegraphics[width=0.35 \textwidth]{V-Zhang-T.pdf}} % 两个\subfloat之间加回车,图片会换行\hspace{2mm}
    \subfloat{
        \label{fig:Qu}
        \includegraphics[width=0.4 \textwidth]{V-Qu-PT.pdf}}

    \subfloat{
        \label{fig:Yang}
        \includegraphics[width=0.75 \textwidth]{V-Yang-T.pdf}}
    \caption{Numerical validations.
        (a)Zhang et al. \cite{Zhang.Wu.ea_2022} of the microchannel heat sink at different mass flow rates for the variation of the bottom surface temperature of the microchannel heat sink.
        (b)Variation of inlet and outlet temperature difference and pressure drop in microchannels of Qu and Mudawar \cite{Qu.Mudawar_2002} at different Reynolds numbers.
        (c)Yang et al. \cite{Yang.Wang.ea_2017} of the pin-fin heat sink with the highest temperature on the bottom surface of the pin-fin heat sink for different pin-fin shapes.}
    \label{fig:Verification}
\end{figure*}

\subsection{Effect of geometric prameters on hydrothermal performance}

In order to explore the influence of the geometric




\subsubsection{The effect of relative rib height}

\subsubsection{Performance analysis}



\begin{table*}[!ht]
    \renewcommand{\arraystretch}{1.5} % 调整行距
    \centering
    \scriptsize
    \caption{Comparison with other solutions}
    \begin{threeparttable}
        \begin{tabular}{m{1.8cm}<{\raggedright}m{2.5cm}<{\raggedright}m{1.5cm}<{\raggedright}m{1.5cm}<{\raggedright}m{1cm}<{\raggedright}m{1.5cm}<{\raggedright}m{1.7cm}<{\raggedright}} \toprule
            Reference & Cooling methods & Heating area ($mm^2$) & Heat power ($W$) & Flow rate ($ml/min$) & Inlet pressure ($KPa$) & Maximum temperature ($^\circ C$) \\ \midrule
            \multirow{2}{*}{Zhang et al. \cite{Zhang.Zhang.ea_2015}} & \multirow{2}{2cm}{parallel cooling microchannels} & \multirow{2}{*}{$22\times 22$} & \multirow{2}{*}{75} & \multirow{2}{*}{58.1} & 330 & 99.52 \\ \cline{6-7}
            & & & & & $0.096^*$ & $55.7^*$ \\[5pt]
            \multirow{2}{*}{Yin et al. \cite{Yin.Li.ea_2019}} & \multirow{2}{2.2cm}{LTCC with embedded metal pillar arrays} & \multirow{2}{*}{$21\times 21$} & \multirow{2}{*}{20} & \multirow{2}{*}{18.85} & 7.12 & 74.85 \\ \cline{6-7}
            & & & & & $0.021^*$ & $57.35^*$ \\[5pt]
            \multirow{2}{*}{Liu et al. \cite{Liu.Jin.ea_2016}} & \multirow{2}{2.2cm}{LTCC with via holes and liquid metal} & \multirow{2}{*}{$10\times 10$} & \multirow{2}{*}{$30^{**}$} & \multirow{2}{*}{70} & \quad - & 83.85 \\ \cline{6-7}                                                                  
            & & & & & $0.138^*$ & $95.74^*$ \\[5pt]
            \multirow{2}{*}{Yu et al. \cite{YU.HAN.ea_2018}} & \multirow{2}{2.2cm}{LTCC with dual-layer spirals microchannels} & \multirow{2}{*}{$2\times 10$} & \multirow{2}{*}{$23^{**}$} & \multirow{2}{*}{45} & 370.7 & 84.85 \\ \cline{6-7}
            & & & & & $0.071^*$ & $55.54^*$\\\bottomrule
        \end{tabular}
        *\,\, The heat sink is MCHS-RPFEM and the coolant is deionized water.\\
        ** Heat flux($W/cm^2$)
    \end{threeparttable}
    \label{tab:Literature-comparison}
\end{table*}


    \input{Section/en00/Section5.tex} 
\end{taggedblock}



% section 加* 表示不显示序号
\section*{Declaration of Competing Interest}
The authors declare that they have no known competing financial interests or personal relationships that could have appeared to influence the work reported in this paper.
\section*{Formatting of funding sources}
This research did not receive any specific grant from funding agencies in the public, commercial, or not-for-profit sectors.

%% The Appendices part is started with the command \appendix;
%% appendix sections are then done as normal sections
%% \appendix

%% 参考文献
\bibliographystyle{elsarticle-num}                                                                          % 加载参考文献样式文件
\bibliography{References}                                                                                   % 参考文献bib库

\end{document}
\endinput
